\documentclass{article}
\usepackage[utf8]{inputenc}

\usepackage[T2A]{fontenc}
\usepackage[utf8]{inputenc}
\usepackage[russian]{babel}

\usepackage{multienum}
\usepackage{geometry}
\usepackage{hyperref}
\hypersetup{
    colorlinks=true,
    linkcolor=blue,
    filecolor=magenta,      
    urlcolor=cyan,
}

\geometry{
    left=1cm,right=1cm,
    top=2cm,bottom=2cm
}

\title{История}
\author{Лисид Лаконский}
\date{March 2023}

\newtheorem{definition}{Определение}

\begin{document}
\raggedright

\setcounter{tocdepth}{5}

\maketitle
\tableofcontents
\pagebreak

\section{Движение декабристов}

\textbf{Декабристы} — участники российского антиправительственного движения, члены различных тайных обществ \textbf{второй половины 1810-х — первой половины 1820-х годов}, организовавшие на Сенатской площади в Петербурге восстание \textbf{26 декабря 1825 года} и получившие название по месяцу восстания.

\hfill

Идеология декабристов \textbf{не была единой}, но в основном была направлена против самодержавия и крепостного права.

\subsection{История движения}

\subsubsection{Союз спасения («общество истинных и верных сынов отечества»)}

В марте 1816 года гвардейские офицеры (Александр Николаевич Муравьёв и Никита Михайлович Муравьёв, капитан Иван Дмитриевич Як\'{у}шкин, Матвей Иванович Муравьёв-Апостол и Сергей Иванович Муравьёв-Апостол, князь Сергей Петрович Трубецкой) образовали политическое тайное общество «Союз спасения» (с 1817 года «общество истинных и верных сынов отечества»). В него входили также князь И. А. Долгоруков, майор М. С. Л\'{у}нин, полковники Ф. Н. Глинка, П. И. П\'{е}стель и другие.

\paragraph{Устав: «трудиться на общую пользу, препятствовать всякому злу», скрытая цель — представительное правление}

Устав общества, в котором была выражена его цель «трудиться на общую пользу, препятствовать всякому злу», был составлен в 1817 году. Скрытую цель общества составляло введение в России \textbf{представительного правления}.

\paragraph{Разногласия и роспуск: Якушкин хочет убивать, решение создать более многочисленную организацию}

Предложение И. Д. Як\'{у}шкина осуществить цареубийство во время пребывания императорского двора в Москве вызвало осенью 1817 года разногласия среди членов организации. Большинство отвергло эту идею. Было решено, распустив общество, создать на его основе более многочисленную организацию, которая могла бы повлиять на общественное мнение.

\subsubsection{Союз благоденствия}

В январе 1818 года был образован «союз благоденствия». О существовании этой формально тайной организации было достаточно широко известно. В её рядах насчитывалось около двухсот человек.

\paragraph{Цели и деятельность: распространение либеральных и гуманистических идей; конституционное правление и ликвидация крепостничества}

Целью «союза благоденствия» провозглашалось широкое распространение либеральных и гуманистических идей, помощь правительству в благих начинаниях и смягчение участи крепостных. Скрытая цель была известна лишь немногочисленным учредителям союза; она заключалась в \textbf{установлении конституционного правления и ликвидации крепостничества}.

\paragraph{Содержание идей: республика, отвержение идеи цареубийства}

На совещании в Петербурге в январе 1820 года при обсуждении будущей формы правления все участники \textbf{высказались за установление республики}. Вместе с тем были \textbf{отвергнуты идея цареубийства и идея о временном правительстве с диктаторскими полномочиями, предложенной П. И. Пестелем}.

\paragraph{Зеленая книга; император не признавал сколь-либо значительного политического значения; некоторые меры предосторожности}

Устав общества, так называемая \textbf{«зеленая книга»}, был известен самому императору Александру I. Но он не признавал в этом обществе сколь-либо значительного политического значения.

Доклад командира гвардейского корпуса об этом тайном обществе и даже записка генерал-адъютанта А. Х. Бенкендорфа, в которой сведения о тайных обществах изложены были с возможной полнотой и с наименованием главнейших деятелей, также осталась без последствий.

Приняты были только некоторые меры предосторожности: в 1821 году сделано распоряжение об устройстве военной полиции при гвардейском корпусе; 1 августа 1822 года последовало высочайшее повеление о закрытии масонских лож и вообще тайных обществ.

\paragraph{Съезд, разногласия и роспуск}

В январе 1821 года в Москве был созван съезд депутатов от разных отделов «союза благоденствия». На нём из-за обострившихся разногласий и принятых властями мер было решено распустить общество. В действительности предполагалось общество закрыть временно для того, чтобы отсеять и ненадёжных, и слишком радикальных его членов, а затем воссоздать его в более узком составе.

\subsubsection{Южное общество}

Вскоре возникли сразу две крупные революционные организации: Южное общество в Киеве и Северное общество в Петербурге. Более революционное и решительное Южное общество возглавил Павел Иванович Пестель, Северное, чьи установки считались более умеренными, — Никита Михайлович Муравьёв.

\paragraph{Южное общество: исключительно офицеры, строгая дисциплина, военный переворот, устав}

В марте 1821 года по инициативе П. И. Пестеля Тульчинская управа «союз благоденствия» восстановила тайное общество под названием «южное общество». В общество привлекались исключительно офицеры, и в нём соблюдалась строгая дисциплина. Предполагалось установить республиканский строй путем военного переворота. Политической программой Южного общества стала «Русская правда» Пестеля, принятая на съезде в Киеве в 1823 году. Возглавляла общество Директория в составе Павела Ивановича Пестеля, Алексея Петровича Юшневского и Сергея Ивановича Муравьёва-Апостола. 

\paragraph{Опора движения — армия, план — вынудить императора отречься}

Южное общество признало опорой движения армию, считая её решающей силой революционного переворота. Члены общества намеревались взять власть в столице, вынудив императора отречься.

\paragraph{Славянский союз — цепные псы}

Во 2-й армии независимо от деятельности южного общества возникло ещё одно общество — «славянский союз», выступавшее за демократическую федерацию всех славянских народов. Оформившееся окончательно в начале 1825 года, оно уже летом 1825 года примкнуло к Южному обществу в качестве славянской управы. Между членами этого общества было много предприимчивых людей и противников правила не спешить. Сергей Иванович Муравьёв-Апостол называл их «цепными бешеными собаками».

\paragraph{Переговоры с польским патриотическим обществам — не привели к сближению и совместным действиями}

В 1823-1825 годах велись переговоры Южного общества с польским Патриотическим обществом о возможности совместного выступления против царизма. Однако они не привели к идеологическому сближению и совместным действиям.

\paragraph{Неудачные переговоры с северным обществом}

Велись также переговоры с Северным обществом декабристов о совместных действиях. Соглашению об объединении препятствовали радикализм и диктаторские амбиции лидера «южан» Пестеля, которых опасались «северяне».

\paragraph{О ведании замыслов южного общества императором и правительством}

В то время, как Южное общество готовилось к решительным действиям в 1826 году, замыслы его были открыты правительству. Ещё до выезда императора Александра I в Таганрог, летом 1825 года, графом Аракчеевым были получены сведения о заговоре. Однако император не принимал никаких радикальных мер и лишь приказывал продолжать шпионскую деятельность, чтобы выявить полный список членов тайного общества.

\subsubsection{Северное общество}

Северное общество образовалось в Петербурге в 1822 году из двух декабристских групп во главе с Никитой Михайловичем Муравьёвым и Николаем Ивановичем Тургеневым. Руководящим органом являлась Верховная дума из трёх человек (первоначально Н. М. Муравьёв, Н. И. Тургенев и Евгений Петрович Оболенский, позже — Сергей Петрович Трубецкой, Кондратий Федорович Рылеев и Александр Александрович Бестужев-Марлинский).

\paragraph{Конституция Н. М. Муравьёва, существование радикального крыла}

Программным документом «северян» была Конституция Н. М. Муравьёва. Северное общество по целям было умереннее Южного, однако влиятельное радикальное крыло (К. Ф. Рылеев, А. А. Бестужев, Е. П. Оболенский, Иван Иванович Пущин) разделяло положения «Русской правды» П. И. Пестеля.

\paragraph{О разногласиях в обществе}

Краевед Якутии Николай Семёнович Щукин в очерке «Александр Бестужев в Якутске» приводит высказывание последнего: «…целью нашего заговора было изменение правительства, одни желали республику по образу США; другие конституционного царя, как в Англии; третьи желали, сами не зная чего, но пропагандировали чужие мысли. Этих людей мы называли руками, солдатами и принимали их в общество только для числа. Главою петербургского заговора был Рылеев».

\paragraph{О боевом течении в обществе}

Академик Николай Михайлович Дружинин в книге «Декабрист Никита Муравьёв» указывает на существующие в Северном обществе разногласия между Н. Муравьёвым и К. Рылеевым и говорит о возникновении в Северном обществе боевого течения, группировавшегося вокруг Рылеева. О политических взглядах участников этого течения Н. М. Дружинин пишет, что оно «стоит на иных социально-политических позициях, чем Никита Муравьёв. Это, прежде всего — убеждённые республиканцы». 

\paragraph{О рылеевской группе в обществе}

Академик М. В. Нечкина говорит о наличии «рылеевской группы» и делает следующий вывод: «Группа Рылеева—Бестужева—Оболенского и вынесла на себе восстание 14 декабря: она явилась тем коллективом людей, без деятельности которого выступления на Сенатской площади просто не произошло бы…»

\subsection{Программные документы}

Составленные декабристами программные документы выявляют глубокие идейные противоречия в их среде. Общее заключалось только в сохранении принципа помещичьего землевладения. Таким образом, не очень понятно, какая именно программа стала бы осуществляться в случае успеха движения.

\subsubsection{Конституция Н. М. Муравьёва — федеративная монархия, отмена крепостного права}

Проект конституции Северного общества предусматривал образование федеративной монархии в составе 14 держав и 2 областей. Предполагалась также отмена крепостного права на условиях наделения крестьян землёй из расчёта 2 десятин на двор, то есть закреплялось крупное помещичье землевладение.

\subsubsection{«Русская правда» П. И. Пестеля — централизованная власть, республика, оставление земли в общинной собственности}

Документ П. И. Пестеля самым коренным образом отличается от программных установок Северного общества. Во-первых, Пестель видел Россию единой и неделимой с сильной централизованной властью. Во-вторых, страна должна была стать республикой. В-третьих, полковник считал, что предназначавшуюся для крестьян землю не следует разделять по дворам, а необходимо оставить в общинной собственности.

\subsubsection{«Манифест к русскому народу» С. П. Трубецкого}

Однако восстание на Сенатской площади проходило под третьим программным документом, который был составлен прямо накануне. Цель восстания состояла в том, чтобы Сенат утвердил этот документ, названный «Манифестом к русскому народу».

\paragraph{Об авторах: Штейнгель, Бестужев, Трубецкой, Рылеев}

Уничтоженную после восстания вводную часть манифеста отдельно друг от друга составляли барон В. И. Штейнгель и Н. А. Бестужев, основную часть — совместно С. П. Трубецкой и К. Ф. Рылеев. Единого экземпляра манифеста сделано не было.

\paragraph{О требованиях: отмена крепостного права, отмена подушной подачи, передача власти временной диктатуре}

Согласно манифесту, Сенат должен был объявить ряд свобод (в том числе отменить крепостное право, при этом вопрос о наделении крестьян землёй не ставился), отменить подушную подать, отправить в отставку «всех без изъятия нижних чинов, прослуживших 15 лет», после чего передать высшую власть временной диктатуре («правлению») в составе 4—5 человек.

\paragraph{О действиях диктаторов после получения власти: разработка плана выборов, образование местного самоуправление, суды присяжных и роспуск армии}

Диктаторы должны были разработать порядок выборов в представительный орган с функциями учредительного собрания. Не дожидаясь созыва упомянутого представительного органа, диктаторам следовало образовать органы местного самоуправления от волостного до губернского уровня вместо прежних чиновников, создать «внутреннюю народную стражу» вместо полиции, образовать суды присяжных и распустить постоянную армию.

\subsection{Восстание}

\subsubsection{Восстание на Сенатской площади}

Среди этих тревожных обстоятельств стали обнаруживаться всё яснее нити заговора, покрывшего, как сетью, почти всю Российскую империю. Генерал-адъютант барон И. И. Дибич, как начальник Главного штаба, принял на себя исполнение необходимых распоряжений; он отправил в Тульчин генерал-адъютанта Чернышёва для ареста главнейших деятелей Южного общества. Между тем в Петербурге члены Северного общества решились воспользоваться междуцарствием для достижения своей цели водворения республики при помощи военного мятежа.

\paragraph{План действий — внушить сомнение гвардии, вынудить императора}

Отречение от престола цесаревича Константина и новая присяга при восшествии на престол императора Николая признаны были заговорщиками удобным случаем для открытого восстания. Чтобы избежать разномыслия, постоянно замедлявшего действия общества, Рылеев, князь Оболенский, Александр Бестужев и другие назначили князя Трубецкого диктатором. План Трубецкого состоял в том, чтобы внушить гвардии сомнение в отречении царевича и вести первый отказавшийся от присяги полк к другому полку, увлекая постепенно за собой войска, а потом, собрав их вместе, объявить солдатам, будто бы есть завещание почившего императора — убавить срок службы нижним чинам и что надобно требовать, чтобы завещание это было исполнено, но на одни слова не полагаться, а утвердиться крепко и не расходиться.

Мятежники были убеждены, что если солдатам честно рассказать о целях восстания, то их никто не поддержит. Трубецкой был уверен, что полки на полки не пойдут, что в России не может возгореться междоусобие и что сам государь не захочет кровопролития и согласится отказаться от самодержавной власти.

\paragraph{Начало и конец восстания}

26 декабря 1825 началось восстание, которое было в тот же день подавлено (расстреляно картечью). По данным чиновника по особым поручениям С. Н. Корсакова, в этот день погибло 1271 человек. 

\subsubsection{Восстание Черниговского полка}

На юге дело также не обошлось без вооружённого восстания.

Шесть рот Черниговского полка освободили арестованного Сергея Муравьёва-Апостола; но 15 января 1826 года были настигнуты отрядом гусар с конной артиллерией. Муравьёв приказал идти на них без выстрела, надеясь на переход правительственных войск на сторону восставших, но этого не случилось. Артиллерия дала залп картечью, в рядах Черниговского полка возникло замешательство, и солдаты сложили оружие. Раненый Муравьёв был арестован.

\subsection{Суд и наказание}

\end{document}